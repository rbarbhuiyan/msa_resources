\documentclass[12pt]{article}
\usepackage[margin=2cm]{geometry}
\usepackage{mathpazo}
\usepackage{parskip}
\usepackage{hyperref}
\usepackage[super]{nth}
\usepackage{enumerate}
\usepackage{csquotes}
\usepackage{graphicx}
%\usepackage{fancyref}
\usepackage{titlesec}
\usepackage{transparent}
\usepackage{draftwatermark}
\usepackage[font=small,skip=-10pt]{caption}

\SetWatermarkText{\transparent{0.1}\includegraphics[angle=45]{ISoc_symbol_logo.png}}

\newcommand\invisiblesubsection[1]{%
  \refstepcounter{subsection}%
  \addcontentsline{toc}{subsection}{\protect\numberline{\thesubsection}#1}%
  \subsectionmark{#1}}

\newcommand{\nocontentsline}[3]{}
\newcommand{\tocless}[2]{\bgroup\let\addcontentsline=\nocontentsline#1{#2}\egroup}

\titleformat{\subsubsection}
  {\normalfont}{\thesubsubsection}{1em}{}

\setcounter{secnumdepth}{3}

\begin{document}
\title{\center{University of Sussex:\\
Islamic Society - Constitution}}
\author{Ridwan Barbhuiyan [r.barbhuiyan@sussex.ac.uk]}
\date{}
\maketitle

\vfill
\begin{center}
\includegraphics{ISoc_banner_green.png}
\end{center}


\newpage
\setcounter{tocdepth}{2}
\tableofcontents
\newpage

\begin{center}
\emph{In the Name of Allah, the Most Compassionate, the Most Merciful.}\\
\end{center}
\section*{Preamble}
The society's formal title, ``University of Sussex Islamic Society'', shall hereafter be referred to as ``ISoc''; the University of Sussex's Students' Union shall hereafter be referred to as the ``SU''; the University of Sussex itself shall hereafter be referred to as the ``US''; and the US building at Falmer House known as the ``Muslim Prayer Facility''/``Muslim Prayer Room''/``Mosque'' shall hereafter be referred to as the ``Masjid''.\\

ISoc mailing address: \href{mailto:isoc@ussu.sussex.ac.uk}{isoc@ussu.sussex.ac.uk}\\
ISoc SU website: \href{www.sussexstudent.com/organisation/isoc}{www.sussexstudent.com/organisation/isoc}\\
ISoc Facebook: \href{www.facebook.com/groups/usisoc}{www.facebook.com/groups/usisoc}\\
ISoc Twitter: \href{http://www.twitter.com/usisoc}{@usisoc}\\
ISoc YouTube channel: \href{www.youtube.com/user/SussexIslamicSociety}{www.youtube.com/user/SussexIslamicSociety}\\


\section{Mission, Aims, and Objectives}

\subsection{Mission Statement}
The ISoc strives to promote the teachings of the Qur'an\footnote{The Holy Book of Muslims.} and Sunnah\footnote{The way of the Prophet Muhammad (peace be upon him/PBUH).}, and cater for its members to fulfil their obligations (according to Islam\footnote{The religion of Muslims -- a body of beliefs/principles said to have been ``completed'' for us (mankind) in the \nth{7} century (Common Era), with the fundamental belief that ``There is no god but the one true god, Allah, and that Muhammad (PBUH) is Allah's final messenger.'' also known as the Shahadah (``Declaration of Faith'').}) to Allah , as well as to facilitate their (and others') learning and understanding of Islam.

\subsection{Aims and Objectives}

\begin{displayquote}
\begin{enumerate}[a.]
\item To strive to uphold the teachings of the Qur'an and Sunnah;
\item To facilitate a suitable place of worship where Muslim\footnote{A follower of Islam (somebody who sincerely professes the Shahadah, by definition, is considered a Muslim).} students, staff, Full Members, Associate Members, and non-Members can offer their five daily prayers (Salaah);
\item To organise and hold the Friday congregational prayer (Jumu'ah);
\item To make suitable arrangements for Ramadan\footnote{The ninth month of the lunar-based Islamic calendar, in which Muslims observe fasting.};
\item To make suitable arrangements for the two holy festivals of Islam -- Eid-ul-Fitr\footnote{The Festival of the Breaking of the Fast.} and Eid-ul-Adha\footnote{The Festival of the Sacrifice.};
\item To strengthen the unity of Muslims in ISoc (as well as at the University of Sussex and in the local area of Brighton \& Hove) and cater for their general wellbeing, via regular sporting and social events;
\item To promote and facilitate the educating of our members about Islam via regular talks, workshops, and education circles;
\item To promote and facilitate both understanding and dialogue between Muslims and non-Muslims via outreach, events (both independent and joint), an ``Islamic Awareness Week'' event, and participating at Students' Union events;
\item To encourage and facilitate the involvement of members in bettering the ISoc, the US, the local area of Brighton \& Hove, and society as a whole, via outreach and community volunteering;
\item To encourage and facilitate the involvement of members in the organisation and running of the society.
\end{enumerate}
\end{displayquote}

\section{Membership, Enrolment, and Withdrawals/Ejections}

\subsection{Membership Types}
ISoc permits three types of Membership (based on SU regulations):
\subsubsection{Ordinary Membership -- Subscribed to the ISoc mailing list, able to attend any and all University of Sussex-exclusive events, participate in official ISoc votes (such as the elections for a new ISoc committee at the Annual General Meeting or ratifying a new/amended constitution at an Open General Meeting), as well as run for all of the committee positions (including the Executive Committee positions [see the relevant section below], Treasurer, and Secretary);}
\label{subsubsec:membership}
\subsubsection{Associate Membership -- Subscribed to the ISoc mailing list, as well as able to attend any and all University of Sussex-exclusive events.}

Ordinary Membership does not require any registration fee i.e. joining ISoc is free. However, for Associate Membership, a fee may be required depending on the status of the Associate Member and according to SU policy and regulations i.e. non-students are required to pay (at the time of publication) \pounds 30, \pounds 15 for students/unwaged, whilst University of Brighton students may become Associate Members for free.

\subsection{Eligibility}
\subsubsection{All Students that are Full Members of the SU are eligible for Ordinary Membership;}
\subsubsection{All others (staff, students that aren't Full Members, Brighton students, and members of the general public) are eligible for Associate Membership.}

\subsection{Registration}
In order to become a member of ISoc:
\subsubsection{Ordinary Member candidates can join ISoc directly via the SU website (as is default for all SU societies);}
\subsubsection{All member candidates (including Ordinary Member candidates), who wish to join, can submit their name and email address to the Secretary, who will in turn submit the information to the SU's Societies Co-ordinator, who will then register the members themselves (in the appropriate category of ``Ordinary Members'' or ``Associate Members'').}

\section{Organisational Structure}
\label{sec:structure}
The ISoc Committee organisational structure is headed by a centralised authority, in the form of the Executive Committee (see \S\ref{sec:EC}), and is arranged as shown in the Fig. \ref{fig:structure}:

\tocless\invisiblesubsection{}
\setcounter{subsection}{0}

\subsubsection{The length of term for ISoc is from April to March of the following year, with a two-week transition period between the outgoing and incoming committee to take place at the start of each ISoc Committee's term, with an official Handover session i.e. the new ISoc committee will begin their term in April, but for the first two term-time weeks of their term, the previous ISoc committee is obliged to help the new committee with the transition, including the handover of any relevant documents/guidelines, contacts, financial accounts, as well completing and submitting the official SU Handover Form;}
\label{subsubsec:handover}
\subsubsection{Should any Committee Members (CMs) anticipate their absence from campus/the US for a duration i.e. travelling abroad, time off due to illness, etc., they should notify the President as soon as possible, so that sufficient cover may be arranged, as well as aid the President in this process;}
\subsubsection{That all Committee Members (CM) are to report to and check things with the President first before carrying out any action, unless it is within the remit of their Role (see relevant section below) or has been specified otherwise by the President;}
\subsubsection{The ISoc committee is a team -- the whole committee needs to work together as a team, and each CM (including the EC) should seek to support their fellow CMs and aid them in their efforts to uphold their responsibilities and better the society. That is to say that, if a particular CM is facing difficulties with their position (be it for reasons that are personal, financial, academic, social, etc.), other CMs should strive to aid them in their responsibilities without assuming authority of the effected CM i.e. if the Treasurer needs help with their role, whoever's helping should not make decisions without the Treasurer's approval;}
\subsubsection{Should CMs fail to uphold their official responsibilities/duties, or behave in a manner contrary to Islamic, US, and SU beliefs/policies whilst in an official capacity, that leads to discord within the Committee and hinders the ISoc in achieving its objectives:}
\begin{displayquote}
\begin{enumerate}[a.]
\item They are to receive a verbal/written warning from the President (the President shall receive theirs from the VP) outlining said failings and suggesting solutions/improvements for the CM in question to implement;
\item Should such failings persist after the aforementioned warning, a tribunal is to be arranged comprising two of the EC and one other CM selected at random (all in the tribunal must not be the CM in question), to decide whether or not the CM in question is to remain on the committee or not, after which the necessary steps must be taken for either outcome i.e. if the CM in question remains, nothing is required, but if the CM in question is ejected, the now-vacant position must be advertised to the Society, with the Committee sharing its responsibilities under the lead of the President;
\item Whatever the outcome, and during the whole term-in-Office, CMs should not make the mistakes/failings of their fellow CMs public affairs that would lead to their public embarrassment/humiliation i.e. any disputes/failings amongst CMs should remain private and within the parties involved.
\end{enumerate}
\end{displayquote}

Note: Each Officer may convene their own sub-committees in relation to their stipulated roles, but will remain responsible for the function of their own Offices.

\begin{figure}[h]
\centerline{\includegraphics[width=0.95\textwidth]{structure.png}}
\caption{The ISoc Committee's Organisational Structure.}
\label{fig:structure}
\end{figure}

\subsection{Executive Committee (EC)}
\label{sec:EC}

\subsubsection{The EC comprises the President, Vice President (VP), and Deputy Vice-President (DVP), and places the President as the primary authority within it, the VP as secondary, and the DVP as tertiary;}
\subsubsection{The VP or DVP will temporarily assume authority of the EC (and thus the ISoc committee) in the unavailability of the rank above them i.e. if the President is unavailable for whatever reason during some decision process or event, leadership will temporarily defer to the VP. As such, when the EC is referred to throughout this document, the same chain-of-command (President $>$ Vice President $>$ Deputy Vice President) is to be assumed i.e. ``check things with the EC first'' would mean to check things with the President first, and if the President is absent, then the Vice President, and so forth;}
\subsubsection{The EC should have \emph{at least} one male and one female member, and so this should be emphasised during advertisement of the aforementioned positions. As such, the highest-ranking member of the EC, for each of the sexes, shall also assume the secondary role and title of ``Brothers/Sisters' Representative'' i.e. if there is a female President, female VP, and male DVP, the President would also be the ``Sisters' Representative'', and the DVP the ``Brothers' Representative'' (see below for role details).}\label{subsec:equality}

\subsection{ISoc Committee Roles}
The ISoc Committee Roles and their respective responsibilities are as follows:

\subsection{President}
\subsubsection{Responsible for the management of the society, with particular regards to continuing ISoc's mission, as well as maintaining and achieving its aims and objectives, respectively;}
\subsubsection{Chairs the ISoc Committee Meetings (ICM);}
\subsubsection{Represents the society at the SU, US, and external meetings;}
\subsubsection{Delegates the above responsibility to the VP/DVP in their place;}
\subsubsection{Assumes the Offices of unfilled Committee Roles (or temporarily absent CMs), or may delegate these Offices to other members of the EC, until they are no longer unfilled (or absent), and shares the responsibilities with the rest of the Committee;}
\label{subsubsec:unfulfilled}
\subsubsection{Cannot take part in Committee Votes unless, in the event of a voting stalemate (within the committee), they (the President) shall have the casting vote.}
\hspace{1pt}

\subsection{Vice-President (VP) and Deputy Vice-President (DVP)}
Note: As implied in \S\ref{subsec:equality}, \emph{at least} one of these two positions must be filled by a Full Member who is of the opposite sex to that of the President.
\subsubsection{In the absence of the President/VP, shall act as the President/VP and assume their role and respective responsibilities;}
\subsubsection{Supports the President in representing the ISoc;}
\subsubsection{Responsible for overseeing key ISoc events and activities e.g. Freshers' Week, Discover Islam Week, Charity Week, etc.;}
\subsubsection{Both the VP and DVP must ensure that the views of all members are given representation as far as possible, as part of their role as Brothers'/Sisters' Representative (more information below).}
\hspace{1pt}

\subsection{General Secretary}
\subsubsection{Responsible for organising the ICMs and setting out their agenda i.e. collate all points to be discussed by each CM for the following ICM and release it to all CMs \emph{at least} 24 hours before said ICM;}
\subsubsection{In the event of the EC's absence, they shall chair the relevant ICM;}
\subsubsection{Takes the minutes of each ICM and release them to all CMs no more than 24 hours after each ICM;}
\subsubsection{Oversees the implementation of points agreed in ICMs i.e. produce a clear list of all the actions decided upon in an ICM (with whose action it is) included in each ICM's minutes;}
\subsubsection{Responsible for internal and external ISoc communication i.e. receive emails addressed to the ISoc mailing address and forward them to the relevant CM, as well as contact external contacts for general ISoc issues;}
\subsubsection{Maintains and manages the ISoc membership list and ISoc Facebook page;}
\subsubsection{Responsible (along with the Publicity Officer) for informing members about ISoc activities via email, the Facebook page, and other relevant media (e.g. Twitter);}
\subsubsection{Responsible (along with the Publicity Officer) for ensuring the monthly prayer timetable is distributed to members via email, the Facebook page, and other relevant media (e.g. Twitter) before the corresponding month begins;}
\subsubsection{Books appropriate rooms and venues for all ISoc meetings, activities, and events.}
\hspace{1pt}

\subsection{Treasurer}
\subsubsection{Is a signatory and liaison to the ISoc bank account/s;}
\subsubsection{Presents the annual account/s financial records at the AGM;}\hspace{1pt}
\subsubsection{Responsible for ensuring and supervising the proper financing of ISoc activities, Events, and Socials, including discussing (with the relevant organising CM) the budget required (or estimated) for said activities, Events, and Socials;}
\subsubsection{Reimburses CMs' ISoc expenses on presentation of a valid receipt.}
\hspace{1pt}

\subsection{Outreach Officer}
\subsubsection{Organises regular (at least fortnightly, ideally weekly) ISoc Outreach activities -- activities to share the teachings/beliefs of Islam, such as halaqa\protect\footnote{The Arabic term for ``a gathering with the intent of learning about Islam and theology'' in the context of Islam.}, community volunteering, Da'wah\protect\footnote{The Arabic term for ``inviting others to knowledge of Islam'' in the context of Islam} stalls, etc. -- for both Members and non-Members (Outreach activities can focus on Members, non-Members, or both) that are in accordance with Islamic, US, and SU beliefs/policies;}
\subsubsection{Liaises with the Treasurer regarding budgets for Outreach;}
\subsubsection{Actively seeks out Outreach opportunities within the local and university communities, as well as to collaborate on such activities where relevant and appropriate;}
\subsubsection{Assumes responsibility of Outreach activities i.e. if there is an ISoc Da'wah stall, and a complaint is made against an Outreach volunteer at the stall, the Outreach Officer is to deal with it and resolve the issue as peacefully as possible, in accordance with Islamic, US, and SU policies;}
\subsubsection{Notifies the Publicity Officer of Outreach activities \emph{at least} two weeks prior to their arranged date, in order for the Publicity Officer to produce and advertise the relevant posters/material to advertise said Outreach activities \emph{at least} one week in advance of them;}
\subsubsection{Attends and represents ISoc (or makes suitable and appropriate arrangements for another to go in their place) at religious-related SU/US events, such as the monthly meetings of US Chaplains at the Meeting House.}
\hspace{1pt}

\subsection{Prayer Room Officer (PRO) [x2]}
There are two positions for this role -- one for males and one for females (so that the appropriate needs/wants of both Brothers and Sisters in the Prayer Room are adequately met).
\subsubsection{Responsible for ensuring the Masjid and its facilities (including kitchens and washrooms/bathroom) are kept in good, clean, conditions, including reporting any cleaning or maintenance issues to the relevant persons (Sussex Estate Facilities, hereafter ``SEF'');}
\subsubsection{Ensures the Masjid and its uses conform to US policies;}
\subsubsection{Works alongside US staff and its contractors to carry out any improvements to the Masjid;}
\subsubsection{Maintains and manages an inventory for all ISoc-owned items within the Masjid;}
\subsubsection{Responsible for ensuring there are sufficient and relevant books available to ISoc members and visitors in the Masjid (and including these in the aforementioned inventory);}
\subsubsection{Maintains, manages, and updates the ISoc noticeboards and any ISoc signage in the Masjid, promptly;}
\subsubsection{Responsible for making sure arrangements are in place for Jumu'ah i.e. an appropriate khateeb\protect\footnote{The Arabic term for the person who delivers a sermon, particularly at Friday/Eid Congregational Prayers.}, appropriate spatial arrangements, that the Public Address system is functioning and ready, any and all collection buckets are appropriately labelled and distributed, and to deliver any ISoc-approved announcements (should the khateeb not be willing to do so);}
\begin{displayquote}
\begin{enumerate}[a.]
\item Maintains and manages a list of ISoc-approved khateebs.
\end{enumerate}
\end{displayquote}
\hspace{1pt}

\subsection{Publicity Officer}
\subsubsection{Maintains and manages the ISoc website/s, Facebook, and all other social media platforms that ISoc has an official presence on (e.g. Twitter and YouTube), to publicise ISoc activities and events;}
\subsubsection{Designs booklets, flyers, leaflets, posters, videos, etc. to advertise ISoc activities and Events promptly (as such, the Publicity Officer ought to be at least competent/confident with graphic design software, such as Adobe Illustrator/Photoshop, but ideally also with sound-mixing and video-editing software, such as Audacity and Adobe After Effects/Premiere Pro);}
\subsubsection{Responsible (along with the General Secretary) for informing members about ISoc activities via email, the Facebook page, and other relevant media (e.g. Twitter);}
\subsubsection{Responsible (along with the General Secretary) for ensuring the monthly prayer timetable is distributed to members via email, the Facebook page, and other relevant media (e.g. Twitter) before the corresponding month begins;}
\subsubsection{Responsible for arranging and managing audio/video recording (where permitted) of ISoc events.}
\hspace{1pt}

\subsection{Socials Officer [x2]}
There are two positions open -- one for males and one for females (so that the appropriate needs/wants of both Brothers and Sisters for social wellbeing are adequately met).
\subsubsection{Organises regular (at least monthly) ISoc Socials for both Brothers and Sisters (Socials can be joint or independent, as the Socials Officers please) that are in accordance with Islamic, US, and SU beliefs/policies;}
\subsubsection{Liaises with the Treasurer regarding budgets for Socials;}
\subsubsection{Organises Eid-ul-Fitr and Eid-ul-Adha Events and liaises with the SU Events team regarding these two events;}
\subsubsection{Ensures ISoc Socials take place in a halal\protect\footnote{The Arabic term for ``permissible'' in the context of Islam i.e. according to Shari'ah.} and legal manner;}
\subsubsection{Assumes responsibility of Socials i.e. if there is a meal out, and a complaint is made against an ISoc member at the Social, the Socials Officer/s in charge are to deal with it and resolve the issue as peacefully as possible, in accordance with Islamic, US, and SU policies;}
\subsubsection{Notifies the Publicity Officer of Socials/Events \emph{at least} two weeks prior to their arranged date, in order for the Publicity Officer to produce and advertise the relevant posters/material to advertise said Socials/Events \emph{at least} one week in advance of them.}
\hspace{1pt}

\subsection{Sports Officers [x2]}
There are two positions open -- one for males and one for females (so that the appropriate needs/wants of both Brothers and Sisters, regarding Sports activities, are adequately met).
\subsubsection{Organises regular (at least monthly, ideally fortnightly/weekly) sports activities, including bookings and appropriate advertising/notice to the ISoc (the latter can be arranged with the Publicity Officer if they wish), that are in accordance with Islamic, US, and SU beliefs/policies;}
\subsubsection{Seeks to engage other societies/teams (both within the US and beyond it) in friendly sporting competitions/tournaments;}
\subsubsection{Facilitates for the needs/wants of the ISoc regarding sporting activities if there is appropriate interest i.e. if enough Members to form a volleyball team are interested in forming a volleyball team affiliated to ISoc and competing in, say, intra-mural competitions, the Sports Officers should do so;}
\subsubsection{Ensures ISoc Sports activities take place in a halal and legal manner;}
\subsubsection{Assumes responsibility of ISoc Sports activities i.e. if there is a match, and a complaint is made against an ISoc member at the Sports activities, the Sports Officer/s in charge are to deal with it and resolve the issue as peacefully as possible, in accordance with Islamic, US, and SU policies;}
\subsubsection{Notifies the Publicity Officer of Sports activities \emph{at least} two weeks prior to their arranged date, in order for the Publicity Officer to produce and advertise the relevant posters/material to advertise said Sports activities \emph{at least} one week in advance of said them.}
\hspace{1pt}

\subsection{Brothers'/Sisters' Representatives [x2]}
\subsubsection{Acts as the primary liaison between the ISoc itself and its Committee i.e. the Brothers' Representative is the primary liaison between the Brothers and the Committee, and the Sisters' Representative is the primary liaison between the Sisters and the Committee;}
\subsubsection{Ensures comments, constructive feedback, and welfare issues from the ISoc are taken forward to the ISoc Committee/relevant members of the ISoc Committee;}
\subsubsection{Actively seeks to improve the ISoc (in accordance with Islamic, US, and SU policies) for its Members by being aware of any on-going issues/shortcomings its Members are facing.}
\hspace{1pt}

\section{Meetings}
\subsection{ISoc Committee Meetings (ICMs)}
\subsubsection{The ISoc Committee are to meet regularly (at least once a fortnight, but ideally on a weekly basis) in order to keep the society active and productive;}
\subsubsection{ICMs should be arranged at a time where a maximum number of CMs are able to attend;}
\subsubsection{All CMs are to attend ICMs (unless with a valid reason to not do so, in which they must inform the EC and Secretary as soon as possible);}
\subsubsection{ICMs are to follow an agenda, with each CM to report the outcome of any actions previously tasked to them, as well as to speak their points of discussion, set out in the following order, whilst the Secretary takes minutes:}
\begin{displayquote}
\begin{enumerate}[a.]
\item EC -- President, VP, DVP;
\item Treasurer;
\item Education;
\item Outreach;
\item PRO;
\item Publicity;
\item Socials;
\item Sports;
\item Brothers'/Sisters' Representatives;
\item Any Other Business (AOB).
\end{enumerate}
\end{displayquote}
\subsubsection{ICMs are open to non-committee members of ISoc for the purpose of spectating and learning how the society runs, but they may not participate in ICM votes.}

\subsection{Annual General Meeting (AGM)}
\subsubsection{The AGM is to be held annually, occurring \emph{at least} one week before the SU deadline ($\sim$March);}
\subsubsection{The AGM comprises at least the following (in the same order):}
\begin{displayquote}
\begin{enumerate}[a.]
\item A summary of the ISoc's activities (including Events, Socials, and Sports) during the outgoing Committee's year-in-Office (especially notable accomplishments/achievements by the ISoc) is presented to the attendees;
\item A summary of the ISoc's Annual Finances Report, by the Treasurer, is presented -- the Annual Finances Report itself should contain the opening and closing balance/s of ISoc's bank account/s, a detailed list of ISoc's incomes and expenditures that includes which activities, Events, Socials, Sports, etc. funds were spent on/came from;
\item Any decisions/votes regarding ISoc/amendments to be made to the Constitution (this document) to take place (with only Ordinary Members eligible to vote) provided a valid quorum is present (see below);
\item The results of the ISoc Elections for the new, incoming committee, in the following order (and thus the following term's ISoc Committee):
\begin{enumerate}[i)]
\item Sports Officers;
\item Socials Officers;
\item Publicity Officer;
\item Prayer Room Officer;
\item Outreach Officer;
\item Treasurer;
\item Secretary;
\item Executive Committee:
\begin{enumerate}[1.]
\item Vice-President and Deputy Vice-President;
\item President.
\end{enumerate}
\end{enumerate}
\end{enumerate}
\end{displayquote}
\subsubsection{The AGM is open \emph{only} to Members of the ISoc (that is to say, attendees must be registered with the ISoc as Ordinary or Associate Members);}
\subsubsection{A quorum to ratify decisions made at the AGM is a \emph{minimum} of 30 Ordinary Members. Should any ISoc-related votes take place, only Ordinary Members (not Associate Members) of the ISoc are allowed to vote (as per \S\ref{subsubsec:membership}), be they Muslim or not;}
\subsubsection{Details regarding the AGM (including its Agenda, time, and place) must be published via the ISoc's mailing list, website, Facebook page, and any other relevant social media platforms, \emph{at least} two weeks before the AGM itself.}

\subsection{Open General Meeting (OGM)}
\label{subsec:ogm}
\subsubsection{This may be called by either:}
\begin{displayquote}
\begin{enumerate}[a.]
\item Twenty-five Members with \emph{at least} ten days notice or;
\item Five CMs with \emph{at least} ten days notice.
\end{enumerate}
\end{displayquote}
\subsubsection{Upon receipt of notification of an OGM and its proposed agenda, the General Secretary must then make necessary and suitable arrangements for it and notify all Members via the ISoc's mailing list, website, Facebook page, and any other relevant social media platforms, \emph{at least} one week before the OGM itself;}
\subsubsection{A chair is to be elected at the start of the OGM;}
\subsubsection{A quorum to ratify decisions at the OGM is a \emph{minimum} of twenty Ordinary Members.}

\section{Elections: Nominations, Results, and Handover}
\subsubsection{The annual ISoc Election polls (for the next year's committee) should open \emph{at least} three business-days before the ISoc AGM and take place online on the official ISoc SU website, in accordance with SU policy;}
\begin{displayquote}
\begin{enumerate}[a.]
\item The online Elections will list each position with its respective candidates, as well as the option to Re-Open Nominations (RON), whereby voters will select their voting options with a ranked format i.e. Candidate 2 receives the highest preference, then RON, then Candidate 1;
\end{enumerate}
\end{displayquote}
\subsubsection{Only Ordinary Members are eligible to run for a committee position i.e. be a candidate, in accordance with SU policy;}
\subsubsection{Only Ordinary Members are eligible to vote in the Elections, in accordance with SU policy;}
\subsubsection{Candidates must nominate themselves for up to two positions and must submit their nomination to the President and General Secretary \emph{at least} one week before the Election polls open (upon which the ISoc will be emailed a confirmed list of candidates) in order to run for their applied positions, along with a manifesto (single PDF file no more than 3MB) and brief description of themselves (no more than 100 words) if they wish (to be hosted on the ISoc SU website; see later);}
\begin{displayquote}
\begin{enumerate}[a.]
\item If a candidate applying for two positions wins the poll for both positions, they can only take up Office as one of those positions, which they must choose at the time of completing the Societies Handover form (from the SU) i.e. when they officially become a part of the ISoc Committee, with the other position automatically going to the candidate with the next highest number of votes in that position;
\end{enumerate}
\end{displayquote}
\subsubsection{Candidates are eligible to campaign, provided that:}
\begin{displayquote}
\begin{enumerate}[a.]
\item Their campaigning is within Islamic, US, and SU beliefs/policies i.e. no form of bribery, smearing of others, etc.;
\item They campaign any time between confirmation email (of candidates) sent out by the ISoc and the polls opening and not outside of these times (which may result in removal as a candidate);
\item All/any campaign materials used are removed/unavailable outside of the campaign time;
\item That no endorsement comes from the ISoc itself (as the ISoc itself must remain independent and neutral towards the candidates and elections) i.e. a candidate may not use ISoc branding (such as the logo or banner) on their campaign materials (be it digital, physical, or anything else) or claim.
\end{enumerate}
\end{displayquote}
\subsubsection{Candidates may use the boards in the Prayer Room (not the official ISoc noticeboards themselves) to display their campaign material, as long as it does not occupy more than an A4 space on each of the Brothers' and Sisters' sides and that they have obtained permission from either the Prayer Room Officer or the President;}
\subsubsection{The ISoc must arrange, host, and publish a webpage on its SU website with the positions listed and the candidates running for each position, along with their brief description and a link to their manifesto (as mentioned earlier). This webpage is to be made publicly available from the moment when the confirmation email (of candidates) is sent out to the ISoc to the moment when the Elections polls close;}
\subsubsection{In the event that a position/s are left vacant, it is the duty of the newly-elected committee to share the responsibilities of this role (with the President leading the decisions, in accordance with \S\ref{subsubsec:unfulfilled}), until somebody is found to fill it;}
\subsubsection{Once the Elections are closed, the results are announced by the President at the AGM, as well as published and distributed via the mailing list, Prayer Room noticeboard, and social media platforms;}
\subsubsection{The newly-elected Committee begins their term in April, with a two-week Handover transition period, as outlined in \S\ref{subsubsec:handover}, with the Societies Handover form completed and submitted;}
\subsubsection{During the Handover, the outgoing Committee must communicate and transfer any and all relevant documents/contacts/files used by the outgoing Committee during their term-in-Office, as well as ensure that the newly elected Committee are fully aware of their responsibilities and how the ISoc operates, as outlined in this Constitution.}
\hspace{1pt}

\section{Constitutional Amendments}
\tocless\invisiblesubsection{}
\setcounter{subsection}{0}

\subsubsection{In order for any amendment (deemed necessary) to be made to the Constitution to be ratified, it must:}
\begin{displayquote}
\begin{enumerate}[a.]
\item First be voted in favour of by a simple majority ($>$1/2) within the ISoc Committee;
\item Provided the first vote (by the Committee) is successful, then be voted in favour of by a supermajority ($>$2/3) of Members present at an OGM or the AGM (provided there is a quorum present);
\begin{enumerate}[i.]
\item If a quorum is not present, the amendment is provisionally accepted until the next OGM/AGM, after which it must be voted on again (first, by the committee, secondly by the Members present at the OGM/AGM, provided there is a quorum present);
\item If an amendment to the Constitution is sought by Members of ISoc outside the Committee, they must call an OGM (following the criteria set out in \S\ref{subsec:ogm} i.e. \emph{at least} 25 Members calling for the OGM with \emph{at least} ten days notice), and may bypass the first criteria i.e. the proposed amendment does not need to be voted in favour of within the ISoc Committee.
\end{enumerate}
\end{enumerate}
\end{displayquote}
\subsubsection{A log of amendments made to the Constitution must be made at the end of this Constitution, including details such as the date the amendment was made, where in the Constitution the amendment itself is (including the previous iteration of point in the Constitution, if it was changed), and the results for both the Committee and AGM/OGM votes (in terms of For and Against numbers/percentages, not identities of voters).}

\section*{Supplementary}
In the event of a dispute as to the meaning of any or all parts of this Constitution, the matter shall be referred to the SU for final adjudication on the matter.

\section*{Log of Amendments to the Constitution}
A log of all amendments to the Constitution should be kept here, detailing the date, by whom the amendment was proposed and their status in the society (if a committee member, including position, or a non-committee member), as well as the results of the vote (including numbers of those who voted For, Against, or chose to Abstain):

\begin{displayquote}
\begin{enumerate}[1.]
\item 2015-09-14:	Constitution written by Ridwan Barbhuiyan (President of ISoc 2015/2016), using the University of Southampton's ISoc's Constitution as a template.
\end{enumerate}
\end{displayquote}

\end{document}